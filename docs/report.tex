\documentclass[journal, a4paper]{IEEEtran}

\usepackage{graphicx}   
\usepackage{url}        
\usepackage{amssymb}
\usepackage{amsmath}    

% Some useful/example abbreviations for writing math
\newcommand{\argmax}{\operatornamewithlimits{argmax}}
\newcommand{\argmin}{\operatornamewithlimits{argmin}}
\newcommand{\x}{\mathbf{x}}
\newcommand{\y}{\mathbf{y}}
\newcommand{\ypred}{\mathbf{\hat y}}
\newcommand{\yp}{{\hat y}}

\newif\ifanonymous
\anonymoustrue

\begin{document}

% Define document title, do NOT write author names for the initial submission
\title{QuoridorRL: solving a two-player strategy game with reinforcement learning}
\ifanonymous
\author{Anonymous Authors}
\else
\author{Nathan Pollet, Rebecca Jaubert, Laura Minkova, Erwan Umlil and Clément Jambon}
\fi
\maketitle

% Write abstract here
\begin{abstract}
	A quick guide to producing the project report. 
	The structure outlined here is a suggestion, but feel free to change if convenient. But you must use this IEEE template. Of course, replace these hints/instructions/examples with your own text. Recall: authors should be anonymous (for the initial submission); page limit of 5 pages, not including references and appendix. 
	
Hint: shared tools like \texttt{http://overleaf.com/} are great tools for collaborating on a multi-author report in \LaTeX. There are also Word templates for this format (\url{https://www.ieee.org/conferences/publishing/templates.html}) if you wish; but you must submit a \texttt{pdf} file. 
\end{abstract}

% Each section begins with a \section{title} command
\section{Introduction}
\label{sec:intro}

In this section, answer the following: 

\begin{itemize}
	\item What aspect (of RL) are you specifically looking at?
	\item Why is it interesting or important?
	\item What are the challenges involved?  
	\item What is known/what has been done by others?
	\item What are the key components of your approach? (What exactly did you do? Did you \ldots design a new environment, study a real-world problem, or a particular agent, or a survey different agents, carry out a theoretical analysis, get new empirical results \ldots) -- be careful to distinguish what you did particular to this project. 
	\item What are your main results? What do they imply?
	\item  What are the specific limitations (could be done next if you had time). 
\end{itemize}

Also: 
		Provide a link to your [anonymous] code\footnote{Something like this: \url{http://anonymouslinktoyourcode.zip}}. 

\section{Background and Related Work}

It is essential to provide sufficient and clear background to your work; targeted towards someone who has attended the course but needs a pedagogical reminder about the parts relevant to your project. Think about this: If you understand something, you should be able to express simply, and in a different way to that which you learned it (in the least: in your own words). 

References are essential; Articles, either published \cite{Astar} or preprint \cite{DeepMindSC2}; book chapters, e.g., Chapter 3 from \cite{Barber}, or even lectures, e.g.,  \cite{Lecture3}, blog posts and code repositories\footnote{A footnote will typically do for a \url{url} such as a GitHub page}. In all cases you \emph{must} properly cite all work that is not your own, including text, figures, results, and code.

Don't hesitate to use and reference equations, but rigorously check that each part of your notation is introduced clearly. For example, Eq.~\eqref{eq:MAP} is a multi-label prediction based on conditional probability $P$, conditioned on input $\x \in \mathbb{R}^2$, with regard to outputs $\y \in \{0,1\}^L$ ($L$ labels in total). 

\begin{equation}
	\label{eq:MAP}
	% Note the example \newcommand s defined above which make it faster to write latex math
	\ypred = \argmax_{\y \in \{0,1\}^L} P(\y|\x)
\end{equation}

\section{The Environment}

Describe and fully detail the MDP which describes your environment(s): state space, action space, reward function, $\gamma$ (if relevant), etc. What are the main challenges this environment poses (for an agent)?

Don't hesitate to use diagrams, figures, and screenshots wherever they are useful. 

Don't forget to expand on the main points you outlined in Section~\ref{sec:intro}.

Detail the main challenges in terms of complexity of the game: large state and branching complexity.


\section{The Agent}

Describe your agent, in answering the following points:
\begin{itemize}
	\item What type(s) of agent(s) do you consider
	\item Why this selection/design (main advantages/disadvantages)
	\item How did you implement/configure/parametrize it 
\end{itemize}

Don't forget to expand on the main points you outlined in Section~\ref{sec:intro}.

\section{Results and Discussion}

Test your agent(s) in the environment(s), show the results, -- and most importantly -- discuss and \emph{interpret} the results. Don't just narrate what you did and observed, but discuss the implications of the results. Method A beats method B -- but why? How? In which contexts?

Don't forget: negative results are also results. Your agent didn't perform as expected? If you can explain why this is just as an important contribution. 

Always discuss limitations, whether observed in your results or suspected in different scenarios. 

Make use of plots, e.g., Fig.~\ref{results_figure}, tables (e.g., Table~\ref{results_table}), etc; anything that illustrates the performance of your agent in the environment under different configurations. Make sure to clearly indicate the parametrization behind each result ($\gamma$, etc.). 

% \begin{figure}[ht]
% 	\centering
% 	\includegraphics[width=0.5\columnwidth]{results_plot.pdf}
% 	\caption{\label{results_figure}Your plots should be as `standalone'/understandable from the labels and the caption as possible: say exactly what the plot is about.}
% \end{figure}

\begin{table}[h!]
	\caption{\label{results_table}Table captions should adequately describe the contents of tables (unlike this one).}
	\centering
	\begin{tabular}{lll}
		\hline
		\textbf{Environment config.} & \textbf{SARSA} & \textbf{Q-Learning}  \\
		\hline
		Simulation 1        & 10             & 15 \\
		Simulation 2        & 12             & 11 \\
		\hline
	\end{tabular}
\end{table}

\section{Conclusions}
	Summarize the project briefly (one paragraph will do). Main outcome, lessons learned, suggestions of hypothetical future work. 
	Reflect upon, but don't needlessly repeat, material from the conclusion. 

% The bibliography:
\begin{thebibliography}{4}

	\bibitem{Barber} % Book
	D.~Barber. Bayesian Reasoning and Machine Learning,
	{\em Cambridge University Press}, 2012.

	\bibitem{Lecture3} % Web document
		In Lecture III - Multi-Output Learning. \textit{INF581 Advanced Machine Learning and Autonomous Agents}, 2022.

	\bibitem{Astar}
	D.~Mena et al. A family of admissible heuristics for A* to perform inference in probabilistic classifier chains.
	{\em Machine Learning}, vol. 106, no. 1, pp 143-169, 2017.

	\bibitem{DeepMindSC2}
	O.~Vinyals et al. StarCraft {II:} {A} New Challenge for Reinforcement Learning.
	\url{https://arxiv.org/abs/1708.04782}, 2017. 

\end{thebibliography}

\newpage
\section*{Appendix}
This is the place to put work that you did but is not essential to understand the paper: additional results and tables, lengthy proofs and derivations, \ldots. Material here does not count towards page limit (but also it will be optional for the reviewer/teacher to work through). 
\end{document}
