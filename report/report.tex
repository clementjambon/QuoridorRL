\documentclass[journal, a4paper]{IEEEtran}

\usepackage{graphicx}   
\usepackage{url}        
\usepackage{amssymb}
\usepackage{amsmath}    

% Some useful/example abbreviations for writing math
\newcommand{\argmax}{\operatornamewithlimits{argmax}}
\newcommand{\argmin}{\operatornamewithlimits{argmin}}
\newcommand{\x}{\mathbf{x}}
\newcommand{\y}{\mathbf{y}}
\newcommand{\ypred}{\mathbf{\hat y}}
\newcommand{\yp}{{\hat y}}

\newif\ifanonymous
\anonymoustrue

\begin{document}

% Define document title, do NOT write author names for the initial submission
\title{QuoridorRL: solving a two-player strategy game with reinforcement learning}
\ifanonymous
\author{Anonymous Authors}
\else
\author{Nathan Pollet, Rebecca Jaubert, Laura Minkova, Erwan Umlil and Clément Jambon}
\fi
\maketitle

% Write abstract here
\begin{abstract}

\end{abstract}

% Each section begins with a \section{title} command
\section{Introduction}
\label{sec:intro}

Insist on the following challenges:
- not much work before
- no human knowledge/pre-existing game data (contrary to Go, Shogi or Chess which have already been widely studied)
- large state space, large branching spaces which bound exhaustive tree search methods to failure

In addition to this paper, we provide the sources of our code at  \footnote{Source: \url{#}}. 

In the following, we first review prior approaches to provide robust and human-level control agents in two-player games with large complexities (section \ref{sec:background}), then we describe our implementation of the environment (section \ref{sec:environment}) and the models we used to solve it (section \ref{sec:models}), followed lastly by our experiments and resulting discussion (section \ref{sec:results}).

\section{Background and Related Work}
\label{sec:background}

\subsection{Complexity}
\label{ssec:complexity}
Describe the state and branching complexity of the game.

\subsection{Training without human knowledge}
\label{ssec:human-knowledge}

As seen in paragraph \ref{ssec:complexity}, \textit{Quoridor} is a game with very high state and branching complexities. Consequently, given the existing computational resources and considering that a decision must be provided by our agent in a moderately realistic amount of time, we cannot perform an exhaustive tree search such as MinMax. Therefore, we turned to an \textit{AlphaGo} approach as presented in \cite{alphago}.

In order to perform Monte Carlo Tree Search over the large complexity space of \textit{Quoridor} and considering the lack of game data contrary to well-known and already analysed, we faced the challenge of training our model without human knowledge. To this extent, we turned to the \textit{"self-play"} approach initially presented in  \textit{AlphaGoZero}\cite{alphagozero} and later expanded to a larger set of games (namely Chess and Shogi) in \textit{AlphaZero}\cite{alphazero}.

\section{Environment}
\label{sec:environment}

As an introduction = environment set up from scratch => coding the came logics + choose consistent state and action spaces representations (and/or) wrapper

\subsection{Game logic}

\subsection{State and action representations}

\subsection{GUI}
We used \textit{Pygame}.


\section{Models}
\label{sec:models}

\subsection{AlphaZero agent}

+ choice of architecture (see appendix)


\section{Results and Discussion}
\label{sec:results}


\section{Conclusions}
\label{sec:conclusion}

\bibliographystyle{plain}
\bibliography{biblio}

\newpage
\section*{Appendix}
This is the place to put work that you did but is not essential to understand the paper: additional results and tables, lengthy proofs and derivations, \ldots. Material here does not count towards page limit (but also it will be optional for the reviewer/teacher to work through). 
\end{document}
